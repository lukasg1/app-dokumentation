\chapter{Einleitung}
\section{Motivation}

\section{Zielsetzung}
Zur Unterstützung soll daher im Rahmen des Moduls „Software Engineering“, welche als Prüfungsleistung die Entwicklung einer mobilen Applikation vorsieht,
eine Ionic-App erstellt werden, welche die Suche und Verwaltung von Brettspielen ermöglicht. Im Kern soll die App über folgende
Funktionalitäten verfügen:
\begin{itemize}
    \item \textbf{Suche nach Spielen:} Eine Suchleiste ermöglicht die Suche nach einer Vielzahl von Brett-, Karten- und Würfelspielen. 
    \item \textbf{Anzeige detaillierter Informationen zu Spielen:} Hierzu gehören unter anderem Name, Hersteller, Erscheinungsjahr, Anzahl der Spieler sowie eine Beschreibung des Spiels. Diese sollen über ein „\ac{API}“ eingespielt werden.
    \item \textbf{Verwaltung von Spielen in einer Wunschliste:} In der Wunschliste können Spiele hinterlegt werden, welche der Nutzer sich zu einem späteren Zeitpunkt kaufen möchte.
    \item \textbf{Verwaltung von Spielen im eigenen Inventar:} Das Inventar bietet eine Übersicht über die Spiele, welche der Nutzer bereits besitzt.
    \item \textbf{Vergabe von Bewertungen zu Spielen:} Dem Nutzer soll es auch ermöglicht werden, eine eigene Bewertung für jedes Spiel abgeben zu können. Diese Funktion soll annhand eines 5-Sterne-Systems realisiert werden.
\end{itemize}
Der Nutzer soll sich zwischen den Seiten der App frei bewegen können. Die App soll dabei eine übersichtliche und intuitive Benutzeroberfläche bieten, welche die Funktionalitäten der App klar darstellt und eine einfache Bedienung ermöglicht.
